\documentclass[10pt,a4paper]{article}

\usepackage[english]{babel}
\usepackage[IL2]{fontenc}
\usepackage[utf8]{inputenc}
\usepackage{graphicx}
\usepackage{array}
\usepackage{url}
\usepackage{hyperref}
\usepackage{amsmath}

\usepackage{cite}

\pagestyle{headings}

\title{Collaborative Recommendation Systems\thanks{Semester project in Engineering Methods year 2024/2025, teacher: Pavol Batalik}}

\author{Andrii Malokhatko\\[2pt]
	{\small Slovak Technical University, Bratislava}\\
	{\small Faculty of Informatics and Information Technologies}\\
	{\small \texttt{xmalokhatko@stuba.sk}}
	}

\date{\small 30th of September 2024}
\newcolumntype{C}[1]{>{\centering\arraybackslash}p{#1}}

\begin{document}

\maketitle

\begin{abstract}

Collaborative Recommendation Systems have played an important role in various industries proving to be an efficient tool to accurately recommend relevant information and enhance user experience using similarities in users behavior. Besides that, such systems have had a profound influence on the industry of recommendation systems and data filtering, particularly in sectors like social media, streaming services,  e-commerce and many others, where personalized suggestions are crucial for user engagement. That's why in this article, I aim to address the following questions:
\begin{enumerate}
\item What are Collaborative Recommendation Systems are?
\item Types of Collaborative Filtering: such as user-based, item-based, and hybrid methods.
\item Problems Collaborative Filtering is facing today  
\item Real-world use cases ranging from product recommendation in e-commerce to content suggestions on social media and streaming platforms.
\end{enumerate}

By addressing these questions, my work serves the purpose of offering basic information on Collaborative Recommendation Systems, while also providing a deeper explanation of some algorithms and structures used in them. I would also like to discuss real-world applications and use cases as well as future directions of collaborative recommendation systems. Finally at the end of this article we will have a look at current challenges and efforts to improve the efficiency and accuracy of Collaborative Filtering Techniques.
\clearpage
\end{abstract}



\section{Introduction}

It is almost impossible to find a person who has never relied on a third party recommendations, either from newspapers, journals, social media, general surveys, or any other sources.  Which proves that people constantly rely on each others opinion. Having said that, Collaborative Recommendation Systems (CRS) use this natural human behavior to identify  users preferences and interests relying on collected recommendations. As a result CRS are highly effective in helping people with tasks like finding trending movies, gripping books, best food recipes, biggest music hits and best-matching products.\cite{10.1155/2009/421425}

\hspace{0.1cm}

That way, the key concept of CRS is leveraging user behavior to analyze patterns in their interactions with content and identify similarities between other users. By doing so, they are able to suggest new items to users with similar tastes. Because of that Collaborative Filtering(CF) is common approach to build recommendation systems as they have proven to be efficient in many applications.
\cite{kumar2015role}

\hspace{0.1cm}

So by term "Collaborative recommendation systems" we understand  recommendation systems that compare if users X and Y  rate n amount of items similarly or have similar behaviours (e.g, buying, watching, listening)  and based of that suggest similar items to both users\cite{10.1155/2009/421425}. Therefore, it can be represented as a table, with U and I being users and items, respectively, and their intersection indicating how each user evaluates individual items.

\hspace{0.1cm}

\begin{table}[h]
    \centering
    %\captionsetup{skip=10pt} % Adds space between the table and the caption
    \begin{tabular}{ |C{1cm}|C{1cm}|C{1cm}|C{1cm}|C{1cm}|C{1cm}| }
    \hline
      & I1 & I2 & I3 & I4 & I5  \\ 
     \hline
     U1 & 0.5 & 0.3 &  &  &  \\  
     \hline
     U2 &  &  & 0.45 &  & 0.4 \\
     \hline
     U3 & 0.5 & 0.7 &  & 0.4 & 0.9 \\
     \hline
     U4 & 0.6 & 0.9 &  & 0.3 & ? \\
     \hline
    \end{tabular}
    \caption{User-Item Interaction Matrix} % Caption at the bottom
    \label{user-item-table} % Label for referencing
\end{table}

\hspace{0.1cm}

From this example, the value of item I5 for U4 can be estimated using a similar user, U3; therefore, U4 would likely rate I5 highly as well. Using this principle CRS are able to produce high-quality predictions regarding users preferred content as it is based on other people with similar tastes. However, there is almost no information on U2 preferences, which presents one of the biggest challenges in collaborative filtering systems. Without corresponding users, such cases lead to a Cold Start problem, which I will address in Section\ref{sec:types_of_collaborative_filtering} \cite{5283866}.

\hspace{0.1cm}

But before we move any further into exploring CRS, I want to clarify the differences between CRS (Collaborative Recommendation Systems) and CF (Collaborative Filtering). CRS refer to systems that use one or multiple CF techniques to come up with precise predictions. While by term CF we understand specific technique such as: User-Based, Item-Based, Memory-Based and Model-Based CF techniques followed by and Hybrid Methods, which focus on finding patterns in user or item similarities to predict one's preferences. Having taken this into account a single CRS can use one or more CF techniques to generate accurate predictions, while CF by itself is closer to an algorithm than to a functional software.


\section{Types of Collaborative filtering}
\label{sec:types_of_collaborative_filtering}
There are several CF Techniques including: Memory-Based, Model-Based approaches as well as Hybrid Techniques. In this section of the article I will separately discuss each technique and make a concise conclusion on where can each technique be effective.

\subsection{Memory-Based Collaborative Filtering}
Memory-based CF algorithms use a user-item interaction matrix to make predictions. They do this by finding so-called "neighbors" — users or items that are similar based on previous interactions. Recommendations are then generated by analyzing these neighbors preferences and behaviors. There are two types of memory-based CF: User-Based and Item-Based:
 
\subsubsection{User-Based Collaborative Filtering}
User-based CF is one the most popular approaches to generating recommendations. This technique predicts user preferences by finding similar users based on their past interactions. The most primitive example would be if User A and User B have similar ratings for many movies, and User B has liked a movie that User A hasn't watched, algorithm is likely to consider that User A would probably enjoy that movie to. \cite{8506344}

The most common and popular approach to calculate similarity between two users is \textbf{Cosine similarity}. In this case user's ratings can be represented as two n-dimensional vectors $\vec{r}_u$ and $\vec{r}_v$:

\[
\text{sim}_{u,v} = \cos(\vec{r}_u, \vec{r}_v) = \frac{\vec{r}_u \cdot \vec{r}_v}{\|\vec{r}_u\| \cdot \|\vec{r}_v\|} =
\]

\[
= \frac{\sum_{i \in I_{uv}} r_{ui} \cdot r_{vi}}{\sqrt{\sum_{i \in I_u} r_{ui}^2} \sqrt{\sum_{i \in I_v} r_{vi}^2}}
\]

\hspace{0.1cm}

here $\text{sim}_{u,v}$ stands for similarity between users u and v, $\|\vec{r}_v\|$ and $\|\vec{r}_u\|$ represent length of each vector, while $r_{ui}$ and $r{vi}$ represent ratings of users u and v on item i. Lastly $I_{u}$ and $I_{v}$ represent set of items rated by user u and v, and $I_{uv}$ items rated by both users respectively. \cite{formulas}

The cosine similarity measure also has an intuitive geometric interpretation. When users’ ratings are represented as vectors 
$\vec{r}_u$ and $\vec{r}_v$ in an $n$-dimensional space (where $n$ is the number of items rated by both users), the similarity between them corresponds to the cosine of the angle $\theta$ between these two vectors. This angle indicates how aligned the two vectors are:

\begin{itemize}
    \item If the angle $\theta$ is close to 0 degrees (or the cosine is close to 1), the vectors are nearly identical in direction, meaning the users have very similar preferences.
    \item If $\theta$ is 90 degrees (cosine is 0), the vectors are orthogonal, indicating no similarity between the users' preferences.
    \item If $\theta$ approaches 180 degrees (cosine is -1), the vectors point in opposite directions, suggesting that the users have opposing preferences.
\end{itemize}

Unfortunately though, this algorithm struggles with scalability and generating recommendations for new users, as their profiles lack sufficient data. A perfect example is shown in the User-Item table (Table \ref{user-item-table}, users 1 and 2) on page \pageref{user-item-table}. \cite{10.1155/2009/421425}

\subsubsection{Item-Based Collaborative Filtering}
Item-based CF were firstly introduced by Amazon in 1998 and is very similar to user-based, but instead it focuses on finding similarities between items, rather then users. That way item-based CF recommend items that are similar to those, user has previously interacted with. A great example would be if user liked Movie A which is highly similar to Movie B, then Movie B would be recommended to user. Additionally, item-based CF is usually easier to scale than user-based CF and works better when there's less information, especially in e-commerce, where the number of items is often much smaller than the number of users. However, item-based CF tends to lack personalization because it doesn’t take into account individual user behavior patterns. \cite{8506344}

\hspace{0.1cm}
\hrule
\hspace{0.1cm}

\subsection{Model-Based Collaborative Filtering}
Model-based CF techniques differ from memory-based methods by using machine learning and statistical models to analyze user-item interactions and make recommendations. Instead of directly comparing users or items. Model-based CF algorithms aim to uncover patterns or features that are not immediately apparent but explain the underlying relationships between users and items. This approach can lead to more personalized and accurate recommendations while addressing issues of scalability that often challenge memory-based methods. For this reason model-based CF is widely employed by large corporations with vast amounts of data, such as YouTube, Amazon and Netflix, because it can handle sparse interaction matrices more effectively.

\hspace{0.1cm}

\subsubsection{Matrix Factorization}
\label{sec:matrix-factorization}
\textbf{Matrix Factorization (MF)} is one popular technique to speed up the computation and increase accuracy in Model-based CF. It reduces a high-dimensional user-item interaction matrix into lower-dimensional matrices representing users and items and then puts up new User-Item matrix, but this time with predicted ratings assigned to each cell. By doing so, MF reveals latent patterns, allowing it to predict missing values (e.g., an unrated movie). Finally this method not only improves the quality of recommendations but also helps in managing the sparsity of the data, which is often a challenge in real-world scenarios.
\cite{matrixfactorization} \cite{koren2009matrix}

\hspace{0.1cm}

One commonly used MF method is known as Singular Value Decomposition (SVD), which has been widely implemented in recommendation systems. This was demonstrated by Koren in their work on the Netflix Prize, where SVD significantly improved prediction accuracy by leveraging patterns within sparse data. This example also perfectly illustrates the challenges opposed by sparse data in bigger corporations, where enormous number of items and users lead to incomplete interaction matrices. \cite{koren2009bellkor}\cite{baker2005singular}.

\hspace{0.1cm}
//
Another widely used model-based technique is \textbf{Deep Learning}. Neural networks can be used to capture more complex, non-linear relationships between users and items. Recent advancements in deep learning models, such as autoencoders, neural collaborative filtering, and convolutional neural networks, have further expanded the capability of model-based CF to handle highly diverse user-item data, delivering personalized recommendations based on deep feature extraction from both user behavior and content \cite{he2017neural}.

In conclusion, model-based CF techniques are highly effective for recommendation tasks requiring scalability, personalization, and adaptability, especially in systems with vast datasets and sparsity issues. However, they require a larger computational effort for training and tuning, making them more resource-intensive than traditional memory-based methods.

\subsection{Hybrid Techniques in Collaborative Filtering} 

\section{Common Problems in Collaborative Filtering}
As we know by now Collaborative Filtering (CF) is effective way to make recommendations, however it's not without challenges as data volume continues to rise CF encounters more and more issues including data sparsity, cold start problem, scalability and other. All this problems drastically affect efficiency, usability and fairness of Collaborative Recommendation Systems (CRS). 

\subsection{Data Sparsity}
Data Sparsity is one of the most common problems in CF. It's because many users interact with only small amount of items, which results in  sparse User-Item (UI) Matrices, making it difficult for CF algorithms to make accurate recommendations, therefore suggested content can be less accurate or even irrelevant. However this problem can be solved with techniques like Matrix-Factorization, which I have covered in section at paragraph \ref{sec:matrix-factorization} \cite{koren2009matrix}

\subsection{Cold Start}
Nowadays the most significant issue in the CF is the cold start problem. It occurs whenever new User or Item is added to UI Matrix. It's because most CF algorithms rely on patterns of user-behavior, so they are not able to accurately generate recommendations for Users or Items without existing ratings. This problem in particular impacts new Users or Items, who may recieve irrelevant recommendations at first. To solve this problem..... \cite{jannach2010recommender}


\bibliography{literatura}
\bibliographystyle{plain}
\end{document}
